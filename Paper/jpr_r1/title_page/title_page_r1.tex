%%%%%%%%%%%%%%%%%%%%%%%%%%%%
% Two Sword Lengths Apart
% 4 May 2015
%%%%%%%%%%%%%%%%%%%%%%%%%%%%

% !Rnw weave = knitr

\documentclass[a4paper]{article}
\usepackage{fullpage}
\usepackage{lscape}
\usepackage[authoryear]{natbib}
\usepackage{setspace}
    \doublespacing
\usepackage{hyperref}
\hypersetup{
    colorlinks,
    citecolor=black,
    filecolor=black,
    linkcolor=cyan,
    urlcolor=cyan
}
\usepackage{booktabs}
\usepackage{dcolumn}
\usepackage{url}
\usepackage{tikz}
\usepackage{todonotes}
\usepackage{verbatim}
\usepackage{endnotes}
\usepackage{graphicx}

\renewcommand*\thetable{\Roman{table}}
\setlength{\belowcaptionskip}{0.5cm}

%%%%%%% Title Page %%%%%%%%%%%%%%%%%%%%%%%%%%%%%%%%%%%%%%%%%%%%
\title{Two Sword Lengths Apart: Credible Commitment Problems and Physical Violence in Democratic National Legislatures}

\author{Christopher Gandrud \\ Hertie School of Governance\footnote{Email: \href{mailto:gandrud@hertie-school.org}{gandrud@hertie-school.org}. Thank you to Gary Cox, Simon Hix, Shirin Rai, Carole Spary, as well as seminar participants at the Hertie School of Governance and Yonsei University for very helpful comments and insights. I would also like to thank Hortense Badarani, Tyler Daveron, Alexander Hall, and Lauren Wallace for research assistance, as well as my students at the LSE for inspiration. An earlier version was circulated under the title ``Two Sword Lengths: Losers' consent and violence in national legislatures''.}}

\begin{document}

\maketitle

\begin{center}
Word Count: 9,812
\end{center}

%%%%%%% Abstract %%%%%%%%%%%%%%%%%%%%%%%%%%%%%%%%%%%%%%%%%%%%
\begin{abstract}
Ideally, national legislatures in democracies should be venues for peacefully resolving conflicts between opposing groups. However, they can become places of physical violence. Such violence can be an indication that countries' legislative institutions are functioning far from the democratic ideal of being venues for peaceful conflict reconciliation. In some cases, such as Ukraine prior to the 2014 outbreak of armed conflicts in the country's east and south, violence can indicate and possibly fuel deeper political divisions. In this first global study of legislative violence, I show that brawls are more likely when legislators find it difficult to credibly commit to follow peaceful bargains. Credible commitment problems are more acute in countries with new democracies and disproportionate electoral outcomes--i.e. when electoral votes for parties do not closely correspond to the legislative seats they are given. I find robust support for this argument using a case study of legislative violence in the antebellum United States Senate and a new global data set. In addition, I find strong evidence that violence is more likely in legislatures with small minority governments.
\end{abstract}


\paragraph{Keywords:} legislatures, violence, credible commitment problems, electoral proportionality, institutional design, majority and minority governments


\end{document}
