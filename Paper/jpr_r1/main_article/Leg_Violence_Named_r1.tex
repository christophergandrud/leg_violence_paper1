%%%%%%%%%%%%%%%%%%%%%%%%%%%%
% Two Sword Lengths Apart
% 4 May 2015
%%%%%%%%%%%%%%%%%%%%%%%%%%%%

% !Rnw weave = knitr

\documentclass[a4paper]{article}\usepackage[]{graphicx}\usepackage[]{color}
%% maxwidth is the original width if it is less than linewidth
%% otherwise use linewidth (to make sure the graphics do not exceed the margin)
\makeatletter
\def\maxwidth{ %
  \ifdim\Gin@nat@width>\linewidth
    \linewidth
  \else
    \Gin@nat@width
  \fi
}
\makeatother

\definecolor{fgcolor}{rgb}{0.345, 0.345, 0.345}
\newcommand{\hlnum}[1]{\textcolor[rgb]{0.686,0.059,0.569}{#1}}%
\newcommand{\hlstr}[1]{\textcolor[rgb]{0.192,0.494,0.8}{#1}}%
\newcommand{\hlcom}[1]{\textcolor[rgb]{0.678,0.584,0.686}{\textit{#1}}}%
\newcommand{\hlopt}[1]{\textcolor[rgb]{0,0,0}{#1}}%
\newcommand{\hlstd}[1]{\textcolor[rgb]{0.345,0.345,0.345}{#1}}%
\newcommand{\hlkwa}[1]{\textcolor[rgb]{0.161,0.373,0.58}{\textbf{#1}}}%
\newcommand{\hlkwb}[1]{\textcolor[rgb]{0.69,0.353,0.396}{#1}}%
\newcommand{\hlkwc}[1]{\textcolor[rgb]{0.333,0.667,0.333}{#1}}%
\newcommand{\hlkwd}[1]{\textcolor[rgb]{0.737,0.353,0.396}{\textbf{#1}}}%

\usepackage{framed}
\makeatletter
\newenvironment{kframe}{%
 \def\at@end@of@kframe{}%
 \ifinner\ifhmode%
  \def\at@end@of@kframe{\end{minipage}}%
  \begin{minipage}{\columnwidth}%
 \fi\fi%
 \def\FrameCommand##1{\hskip\@totalleftmargin \hskip-\fboxsep
 \colorbox{shadecolor}{##1}\hskip-\fboxsep
     % There is no \\@totalrightmargin, so:
     \hskip-\linewidth \hskip-\@totalleftmargin \hskip\columnwidth}%
 \MakeFramed {\advance\hsize-\width
   \@totalleftmargin\z@ \linewidth\hsize
   \@setminipage}}%
 {\par\unskip\endMakeFramed%
 \at@end@of@kframe}
\makeatother

\definecolor{shadecolor}{rgb}{.97, .97, .97}
\definecolor{messagecolor}{rgb}{0, 0, 0}
\definecolor{warningcolor}{rgb}{1, 0, 1}
\definecolor{errorcolor}{rgb}{1, 0, 0}
\newenvironment{knitrout}{}{} % an empty environment to be redefined in TeX

\usepackage{alltt}
\usepackage{fullpage}
\usepackage{lscape}
\usepackage[authoryear]{natbib}
\usepackage{setspace}
    \doublespacing
\usepackage{hyperref}
\hypersetup{
    colorlinks,
    citecolor=black,
    filecolor=black,
    linkcolor=cyan,
    urlcolor=cyan
}
\usepackage{booktabs}
\usepackage{dcolumn}
\usepackage{url}
\usepackage{tikz}
\usepackage{todonotes}
\usepackage{verbatim}
\usepackage{endnotes}
\usepackage{graphicx}

\usepackage[margins]{trackchanges}

\renewcommand*\thetable{\Roman{table}}
\setlength{\belowcaptionskip}{0.5cm}

%%%%%%% Title Page %%%%%%%%%%%%%%%%%%%%%%%%%%%%%%%%%%%%%%%%%%%%
\title{Two Sword Lengths Apart: Credible Commitment Problems and Physical Violence in Democratic National Legislatures}

%\author{Christopher Gandrud \\ Hertie School of Governance\footnote{Email: \href{mailto:gandrud@hertie-school.org}{gandrud@hertie-school.org}. Thank you to Gary Cox, Simon Hix, Shirin Rai, Carole Spary, as well as seminar participants at the Hertie School of Governance and Yonsei University for very helpful comments and insights. I would also like to thank Hortense Badarani, Tyler Daveron, Alexander Hall, and Lauren Wallace for research assistance, as well as my students at the LSE for inspiration. An earlier version was circulated under the title ``Two Sword Lengths: Losers' consent and violence in national legislatures''.}}
\IfFileExists{upquote.sty}{\usepackage{upquote}}{}
\begin{document}

\maketitle

\begin{center}
Word Count: 9,812
\end{center}

%%%%%%% Abstract %%%%%%%%%%%%%%%%%%%%%%%%%%%%%%%%%%%%%%%%%%%%
\begin{abstract}
Ideally, national legislatures in democracies should be venues for peacefully resolving conflicts between opposing groups. However, they can become places of physical violence. Such violence can be an indication that countries' legislative institutions are functioning far from the democratic ideal of being venues for peaceful conflict reconciliation. In some cases, such as Ukraine prior to the 2014 outbreak of armed conflicts in the country's east and south, violence can indicate and possibly fuel deeper political divisions. In this first global study of legislative violence, I show that brawls are more likely when legislators find it difficult to credibly commit to follow peaceful bargains. Credible commitment problems are more acute in countries with new democracies and disproportionate electoral outcomes--i.e. when electoral votes for parties do not closely correspond to the legislative seats they are given. I find robust support for this argument using a case study of legislative violence in the antebellum United States Senate and a new global data set. In addition, I find strong evidence that violence is more likely in legislatures with small minority governments.
\end{abstract}


\paragraph{Keywords:} legislatures, violence, credible commitment problems, electoral proportionality, institutional design, majority and minority governments

\vspace{0.3cm}

%%%%%%% Introduction %%%%%%%%%%%%%%%%%%%%%%%%%%%%%%%%%%%%%%%%%%%%

Though legislators in democracies are often described as `battling' or `fighting' we expect these battles to be in terms of rhetoric and procedural manoeuvres, circumscribed by non-violent rules, that culminate in votes. The outcomes of these contests are then respected by all legislators. This process is central to the democratic ideal  \cite[220]{Schwarzmantel2010}. However, metaphorical battles sometimes become physical fights between legislators.

Many parliaments' histories contain incidents of physical violence between legislators. In 1856 a member of the United States House of Representatives caned a senator unconscious in the Senate chamber \citep{USSenateCanning}. The United Kingdom's House of Commons may be physically designed to prevent violence between members, as the Government and Opposition benches are said to be ``two sword lengths apart" \citep{ParliamentUKSword} so that duels will be fought with words rather than swords. Actual sword fights do not seem to have taken place in the Commons chamber, but brawls did occur in the 1800s \citep[]{ByrneViolence}. Violence in legislatures continues to occur. Recent instances of violence between legislators include fights in South Korea in 2009\footnote{See \url{http://dailym.ai/1rHLruX}. Accessed October 2014} and in Ukraine during the years leading up to the civil war in the country's east.\footnote{For example see \url{http://online.wsj.com/articles/SB10001424052748704471204575209572380473814}. Accessed October 2014} In 2013 a large confrontation happened in the Venezuelan National Assembly when the Assembly President withheld speaking time from legislators who did not recognize the President's victory in a highly contested election.\footnote{See \url{http://edition.cnn.com/2013/04/30/world/americas/venezuela-lawmakers-violence/index.html}. Accessed October 2014.}

Physical violence is a dramatic break from legislative scholars' assumption that compliance with legislative rules of the game is a given. Some work has examined the strategic and expressive choices politicians make when they decide not to participate in the democratic rules of the game, including before elections \citep{wilkinson2006,Beaulieu2008,BeaulieuForthcoming}. In this article I extend this work by advancing a theory of legislative violence in democratic national legislatures and test it using both case study and global-level data. Indeed, this article includes the first cross-country description of violence between legislators. It also provides us with an important window for understanding what makes legislatures work (or not) as democratic institutions for peacefully resolving disputes between opposing parties.\footnote{The process advanced here could also create environments where legislators use other types of less violent--and less easily observable on a global scale--forms of rule breaking.}

I begin by advancing an argument for understanding when legislative violence between members of democratic legislatures is more likely. Though legislators' personalities surely play some part in any given violent incident, the probability of violence is strongly influenced by the wider political environment. Violence is much more likely when there are \emph{credible commitment problems} that incentivize legislators to break peaceful bargains. At least two observable factors are important for affecting the likelihood that commitments will be adhered to: the \emph{proportionality of electoral outcomes}--i.e. when electoral votes for parties do not closely correspond to the legislative seats they are given--and the \emph{age of democracy}. After proposing this argument and discussing a number of key alternative explanations, I begin to empirically examine it with a case study of violence in the antebellum United States Senate. I build on this with a global-level study of contemporary legislative violence. After describing the new data set and regression models, I lay out the evidence from these models that legislative violence is more prevalent in countries with disproportionate electoral outcomes and in new democracies. I also find that violence is more likely in legislatures with small minority governments. I conclude a discussion of the possible implications of these findings for democratic institution designers and directions for future research.

%%%%%%%%%%%%%%%%%%%%%%%%%%%%%%%%%%%%%%%%%%%%%% Previous Research & Framework

\section*{Understanding Legislative Violence}

Legislative violence, like other forms of violent disruption \citep[]{Beaulieu2008,BeaulieuForthcoming,wilkinson2006}, could be used for strategic purposes by both legislative winners and losers. Legislative losers--those who are not in control of the legislative ``procedural cartel''  that sets the rules enabling agenda control \citep{cox2005} and so are not part of the group that has the most control over legislative outcomes\footnote{There is strong evidence that those who control the legislative agenda tend to control what bills are brought up for a vote and in turn what bills are passed \citep[93]{tsebelis2002}.}--may use violence to stall legislation or rule changes they dislike. Winners--those in the procedural cartel--could use violence to prevent losers from utilizing procedures that might constrain their power to control the legislative agenda and therefore legislative policy outcomes. Both winners and losers may use violence to shore up support among their proponents, as a way of expressing dissatisfaction with legislative outcomes, and to publicize issues they and their supporters care about \citep{Spary2013}. Winners may not only use active violence, but also might make a strategic choice not to use their powers, e.g. control of security forces, to prevent or curtail losers' violence with the hope that losers will be publicly discredited.\footnote{Examining the goal-oriented interactions of both those in power and those out of it is in contrast to important strains in the political protest movement literature. Political opportunity structure theories of protest have examined how, largely exogenous, state power \citep{skocpol1979} and state receptiveness to protester demands \citep{mcadam1982,tarrow1989} encourage or discourage protest. In my approach, winners clearly play an important role in causing violence, they may even be violent. However, I do not treat them as exogenous. They are goal-seeking actors engaged in bargaining with other legislators.}

Just because actors can gain a strategic advantage or express their discontent through violence does not mean they will choose to. Though violence may have strategic benefits, it also entails costs. Violent conflict has physical costs. In a number of incidents legislators have been hospitalized or even, as in the case of Charles Sumner discussed below, almost died. Other potential costs include legal penalties and reputational damage.

There are situations where legislators perceive the benefits of violence to be greater than the costs. I argue that violence in democratic legislatures is often precipitated by situations where legislators find it difficult to credibly commit to follow peaceful bargaining outcomes. In these situations the perceived benefits of violence can outweigh the costs. Violence becomes bargaining by other means. \cite{Fearon1995} argued that when actors are not able to make credible commitments, the benefits of violent conflict can outweigh the costs, making violence more likely \cite[see also][]{Powell2006}. If it is difficult to believe that a peaceful bargaining outcome will actually happen because bargained commitments will likely be broken, actors will choose violence to achieve their goals. This logic is applicable to legislatures. In democratic legislatures winners and losers--who may some day become winners--need to be able to credibly commit to not use or remake legislative procedures and policies in their narrow self-interest. They need to commit to limits on their power \citep{riker1982,Gaubatz1996}, especially rules and legislation that are the result of peaceful bargaining processes. If these credible commitments are not possible then legislators may come to believe that disruption and violence are the best ways to achieve their legislative and policy goals despite the costs. What makes legislative credible commitment problems more or less severe?

\subsection*{Observing Situations Where Credible Commitment Problems are Worse}

I focus on two clearly observable factors that influence legislators' credible commitment problems: proportionality of electoral outcomes and age of democracy. Please note that these are likely not the only factors that shape credible commitments, but they are observable in a cross-country study.

\paragraph{Proportional Electoral Outcomes}

Credible commitment problems are generally smaller and therefore violence is less likely when seats in the legislature and legislative resources more generally, such as speaking time and committee appointments, proportionally correspond to voters' support. Control over seats and other legislative resources such as the agenda often directly corresponds to what bills pass the legislature. Incongruity between legislative power and voter support increases legislative credible commitment problems in at least two ways: it (a) creates possibilities for shifts in power from those who benefit from the status quo to beneficiaries of rules that would more closely align legislative resources with voter support and (b) prevents fairness equilibria.

Disproportionate legislatures have the possibility for large and rapid shifts in legislative power from those who benefit from the status quo to those who would benefit from new rules. For example, if an electoral system that creates disproportionate outcomes becomes more proportional, then the winning parties may be likely to change over the course of one election. The new winners could then further alter legislative procedures and legislative outcomes to benefit themselves through control of the legislative agenda. \cite{Powell2004,Powell2006} identified major commitment problems in bargains over issues affecting future bargaining power when there could be large and rapid power shifts. We can apply his logic to legislatures. Temporarily weak legislators--those with disproportionately fewer seats and access to legislative resources under the status quo--who are not in the legislative procedural cartel need to ``buy off'' those that are temporarily strong--the beneficiaries of the status quo in the cartel--in order to avoid the strong changing the rules to further benefit themselves. ``Buying off'' in this context may simply mean agreeing to continue rules that distribute legislative resources away from electoral support at the status quo level. However, because the presently weak have the potential to be much stronger, they are likely to renege on agreements that disproportionately benefit the temporarily strong. The temporarily strong may also have incentives to use disruption and violence to prevent further rule or policy changes that limit their power or changes that distribute resources further in their favor. Importantly, in the absence of credible commitments from the temporarily weak, the \emph{temporarily strong may use pre-emptive violence to stop changes to the rules that distribute legislative resources in closer alignment with electoral support}.

Why do legislatures that have a close correspondence between legislative resources and voter support not create equally large credible commitment problems? Presumably, legislators that would benefit from legislative resources being unmoored from voter support would find it difficult to credibly commit. In other words, why would a legislature where resources were distributed closely according to electoral support create an equilibrium? Rabin's \citeyearpar{Rabin1993} work studying bargaining consequences when actors care about fairness, in addition to material well-being, provides an answer. Because actors care about ``fairness'', they are more likely to maintain commitments (punish defectors), even if it hurts their material well-being, when others are being fair (unfair). Experimental research supports the idea that commitments are more credible if they are fairer \citep{Ellingsen2004} and can even act as an enforcement device for incomplete contracts \citep{Fehr2008}.

There are many ways to conceptualize fairness, but it is reasonable to assume that a close correspondence between votes and seats will generally be viewed as more fair. One form of punishment that could be inflicted on those who break fair--highly proportional--bargains is reputational damage. This reduces legislators' incentives to defect from fair bargains even if they could gain legislative power by increasing disproportionality. As such credible commitment problems are lower and actors can reach a ``fairness equilibrium'' in highly proportional systems.\footnote{Note that if actors did not care about fairness--very high proportionality in this context--then we would expect to see no difference in violence between more and less proportional legislatures. Those who would benefit from more disproportionality could not make credible commitments.}

At what levels of disproportionately would we expect to observe more violence? At most levels there may be credible commitment problems to some degree, because there will be legislators who benefit from an increase in proportionality. It is only when outcomes are close to perfectly proportional that the gains from increasing proportionality are very small or virtually none existent. At the same time in systems with very proportional electoral outcomes it is more likely that all sides identify the electoral outcomes as fair, thus enabling fairness equilibria.\footnote{When there is ambiguity over how `fair' the system is actors may not impose very high costs for breaking the rules.} Because of this, we should not expect a linear relationship between disproportionately and violence. There should instead be a threshold effect where very proportional electoral outcomes will have very low levels of violence due to very small or non-existent credible commitment problems.

Empirical research on the functional form of the relationship between political trust and disproportionality provides some initial evidence for the claim that there could be a threshold effect between disproportionality and legislative violence. \cite{Marien2011} found that there was a curvilinear relationship between proportional electoral outcomes and citizens' political trust. Feelings of political trust were highest with very proportional outcomes, as well as those with disproportional or majoritarian systems. Countries in the middle had the lowest trust. Marien argues that high trust in very proportional systems is caused by high fairness. The fairness effect seems to quickly disappear as we move in the direction of more disproportionate outcomes. Marien argues that high voter trust in very disproportionate countries is caused not by fairness, but by high accountability. Should we expect a curvilinear relationship between proportionality and legislative violence? Probably not. Though accountability may please voters in general, there is little reason to believe that this will ameliorate credible commitment problems between legislators created by unfairness. For these reasons we should expect:

\begin{quote}
    H1: \emph{Countries with very proportional electoral outcomes have fewer incidents of violence.}
\end{quote}

It's important to note that though the exact type of electoral system is ultimately interesting to us from an institutional design point of view, we should not confuse ``the outcome of an electoral system with its mechanics'' \citep[][109]{Golder2005}.

\paragraph{New vs. Old Democracies}

There are a number of reasons that legislators in new democracies are likely to have credible commitment problems. Legislators in new democracies are limited in the information they can gather to predict if ``pretenders to office can expect to reach it, losers can expect to come back'' \citep[][36]{Przeworski1991}. Actual alternations of power allow legislators to gather better information about the credibility of commitments to allow future alternations of power. Increased information about others' abilities to make credible commitments could strengthen the credibility of future commitments, thus reducing violence.

In new democracies there are more opportunities to change the rules, as the status quo has not been fully institutionalized. This can give present winners considerable power to set the rules to their advantage. The first actors to gain power after a transition may be better able to establish rules and policies, that entrench their power and disadvantage others in the future \cite[108]{Saideman2002}. This could lead to credible commitment problems between the temporarily strong that are making the rules during the democratic transition and the temporarily weak.

Furthermore, in the relatively early days of a democracy the legislative party system may be shifting considerably \cite[161]{Mainwaring2007b}, possibly as politicians work out electoral coordination problems \citep{cox1997}. New democracies may have rapidly changing economies and demographics that additionally alter the party system and legislative proportionality. Even if rules proportionally distributed resources when they were created, they may become less proportional as the country changes. On average, these shifts could be much larger in new democracies. As we will see in the antebellum US case below, demographic and party system shifts dramatically altered the proportionality of the distribution of power in the US Senate. Because of the instability of new party systems, there is a greater likelihood that legislatures could become disproportional, leading to credible commitment problems and violence.

Finally, there may be a survivor bias created as democracies age. If legislatures are unable to overcome credible commitment problems in some way, legislative institutions will not peacefully organize bargaining. This could lead to discontent and social unrest both inside and outside of the legislature, possibly resulting in democratic collapse. New democracies unable to overcome credible commitment problems simply may not survive long enough to become old.

For these reasons we should expect to see that:

\begin{quote}
    H2: \emph{Violence is more common in new democracies' legislatures.}
\end{quote}

\section*{Alternative Explanations}

What other factors may contribute to or be alternative explanations of legislative violence?

\paragraph{The Size of the Governing Majority}

The use of legislative procedures may be viewed as more legitimate and therefore worth following simply if there are larger proportions of the parliament supporting them, e.g. if the governing majority is larger. However, the relationship between the size of the governing majority and violence is not clear \emph{ex ante}.

Though far from the only way of thinking about democratic legitimacy \citep{Follesdal2006}, majority rule is a foundational concept of democracy \citep{Dahl1989} and an important component of democratic legitimacy. An extensive literature led by Arend Lijphart makes the argument that perceptions of democratic legitimacy are stronger as the proportion of actors involved in decision-making increases \citeyearpar[]{Lijphart2007}. The more legislators that are involved in parliamentary decision-making, the more likely it is that legislators will view procedures and legislative outcomes as legitimate and worth following.

However, another causal mechanism may be at work if there is a relative lack of violence in legislatures with very large majorities. Perhaps these majorities are so powerful that they can quickly quash legislative disruption before it starts or even prevent serious opposition politicians becoming legislators in the first place. These sorts of actions may be less likely in the types of legislatures that are the focus of this article--democratic legislatures--, though they are certainly not impossible.

We also have good reasons to suspect that very large legislative majorities may increase violence in democratic legislatures. If a parliament has a large hegemonic party, minority politicians may feel marginalized. They may have no way to influence policy-making other than with extreme acts of legislative disruption, like violence.

What about at the other end of the spectrum? Minority governments are often constrained in their ability to pass legislation by themselves. They need to assemble a coalition of opposition politicians in order to pass legislation. Though the official legislative cartel has a minority of the seats, legislation may still require a majority to pass. As such non-government party legislators can influence policy \citep{strom1990minority}. However, this does not necessarily mean that we should expect no difference in the credible commitment problems of minority and majority governments. Though minority governments may be constrained in their ability to pass legislation without the support of other parties, they can still wield considerable agenda setting power, such as by restricting plenary speaking time \citep{tsebelis2002,cox2005,cox2007}. Other legislators may be very inclined to view a minority government's agenda control as unfair and see opportunities to increase their power and shape policy outcomes by changing the rules. Future credible commitments would thus be more difficult to make and violence would be more likely.

\paragraph{Legislative Immunity}

Having laws that outlaw violence and sanction violators of these laws may dissuade physical attacks. In many countries legislators are immune from prosecution or at least arrest in the legislative chamber. Such immunity is often granted in order to prevent the legislature from being harassed and obstructed by the executive or judicial branches of government \citep{Seghetti1984}. Legislators immune from legal consequences may be more likely to physically attack each other.

It is important to note that if legislative violence is created by credible commitment problems, where legislators are not able to commit to peaceful bargains, then they may not necessarily be prevented from using violence due to a lack of immunity. At best, a lack of immunity would make violence more costly thus marginally decreasing credible commitment problems, but not eliminating them. Furthermore, since the ``application of punishment is inherently political'', formal rules may not have any deterrent effect if legislators do not believe they will be applied to them for political reasons \cite[58]{Wolfe2004}.

\paragraph{The Broader Society}

Perhaps broader societal-level factors create contexts where legislative violence is more likely. Some have argued, for instance, that certain regional cultures are less likely to respect democratic institutions. If this is true, then these cultures might be more likely to have legislative violence. The many popular hypotheses about East Asian `Confucian' cultures \citep[]{Inglehart2005, Inglehart2010} and democratic instability are especially relevant for us given the high number of brawls in East Asia, notably in Japan, South Korea, and Taiwan (Figure \ref{leg_map}). One view is that Asian societies have hierarchical and deferential cultures that are incompatible with democracy because authority is valued over self-expression \citep[212-213]{Dalton2005}. It is unclear how this hypothesis would explain the high frequency of legislative violence in Asian democracies. It would seem to actually suggest less violence. Recent empirical evidence has found that Asian societies are in fact not strongly deferential to authority, especially when compared to Western ones \citep{Dalton2005, KimAsianValues2010}. Mostly using Inglehart and Welzel's World Values Survey data, \cite{KimAsianValues2010} finds that East Asian societies have lower respect for authority than non-Asians and South-east Asians. Assuming that societal values are generally congruent with legislators' values, perhaps legislators in East Asian countries are more violent because their members do not respect legislative authorities. Legislative violence in this cultural region would thus simply be the result of the same cause as violence in other societies with low-respect for authority.

Along with culture, various economic and sociological phenomenon may make certain societies more violent than others. For example, an honor culture in the Southern United States is heavily intertwined with economic, racial, and gender issues \citep[]{nisbett1996culture}. This has led to persistently high rates of violence in the South. Perhaps legislators from more violent societies are themselves more likely to use violence in the legislative chamber. Places with higher societal-level violence may have more violent legislators. There are preliminary reasons to be sceptical, however. East Asian countries with many instances of legislative violence tend to have very low levels of societal violence.\footnote{For example, South Korea had a murder rate of 2.6 per 100,000 people in 2010 and Japan's was 0.4 in 2009 \citep{UNMurder2013}.}

An important issue to consider with societal-level explanations of legislative violence is how closely societal-level factors are generalizable to legislators. Legislators are often from relatively privileged segments of society and distinct sub-cultures \citep[408]{Spary2013}. Further work, beyond the scope of this article, is needed to gather a global data set on legislators' cultures and backgrounds to study how they may contribute to legislative violence. Though there may still be challenges untangling the causal story. For example, legislative institutions that create credible commitment problems may exacerbate legislators' conflictual attitudes.

\section*{Case Study: The Caning of Senator Sumner}

This section presents a case study of the caning of US Senator Charles Sumner in 1856. The incident is a useful supplement to the large-n regression analysis below for a number of reasons. First, despite taking place in a Western democracy, where contemporary violence is rare, this case is ``typical'' for the article's theory \citep[299]{Seawright2008}. The United States at the time was a new democracy and the US Senate was very disproportional. As such, it is useful for illustrating and probing our causal mechanisms in detail. Second, the Senate's disproportionality increased over time due to reasons that are largely exogenous of the legislature. This allows us to make a within-case comparison, such that we can explore the causal direction of the relationship between violence and disproportionality. Third, the case is drawn from outside the large-n sample discussed below. This helps us initially explore whether the proposed hypotheses are generalizable to other time periods. Finally, the case illustrates how societal changes can worsen legislative disproportionality, thus increasing credible commitment problems and violence. Given the relatively shorter time span covered by the large-n data set discussed below and the slowness with which most societal variables change, exploring these complex processes in this well-documented case study is more empirically feasible.

The breakdown of the ability to make credible commitments in disproportionate and new democratic legislatures is a key component behind the caning. The development of the US Senate prior to the Civil War, and especially in the 1850s, was preoccupied with the apportionment of pro and anti-slavery senators and how this apportionment was becoming increasingly unfair. Two senators were elected by each state's legislature. From the end of the Revolutionary War there had been more slave states than free states--thus more slave state senators--and a pro-slave Senate veto \cite[151]{Weingast1998}. However, compromises that were made to have at least as many slave states as free states and as many pro as anti-slavery senators led to increased disproportionality. As the 1800s progressed these compromises were undermined by three shocks that were largely exogenous from the national legislature.

These shocks created considerable opportunities for politicians from free states to increase their power by increasing the legislature's proportionality. The shocks were (a) factor endowments in potential new Western states, (b) population growth in free states, and (c) the expansion of the franchise to non-property owning white males. The westward expansion of the United States from the 1850s posed a problem for maintaining commitments to have at least as many senators from slave as free states: there were more potential free states than slave states. This was largely the result of the fact that areas west of eastern Texas\footnote{Admitted as a slave state in 1845.} lacked land conducive to supporting plantations \cite[]{Ramsdell1929,Weingast1998}. In 1800 the North and South had roughly equal populations. However, rapid population increases from the mid-1840s, partially from immigration and industrial expansion, in free states greatly increased the disproportionality of the Senate as each free states' number of senators remained the same despite their increased population \cite[184]{Weingast1998}. Many of these new people were newly eligible to vote. Between the late 1700s and the 1850s all of the states removed almost all of their property ownership voting requirements.

Anti-slavery supporters became very vocal about the increasingly disproportionate allocation of power in the Senate. They regularly used the terms ``slave power'' and ``slavocracy'' to refer to pro-slavery advocates' disproportionate power \citep{richards2000}, especially electoral power in the Senate. For example, a friend of Congressman Horace Mann wrote to him in 1850 that ``I have been astonished for many years to see how the Slave power (not one fiftieth part of the voters) manage to control the whole United States'' \citep[quoted in][6]{Gara1969}. Gara argues that beyond concerns of the morality of slavery, ``the main thrust of [abolitionists'] attack was against slave power'' \citeyearpar[6]{Gara1969}. This concern was partially electorally motivated. Large portions of the electorate, such as white labor--who were afraid of competing with lower cost slave labor--were also concerned with slave power, though not necessarily the immorality of slavery.

The Senate's increasing disproportionality made it difficult for anti-slavery proponents to credibly commit to rules of the chamber that entrenched ``slave power''. If they succeeded in changing the rules so that the chamber was more proportionate their power would increase dramatically. Also, because the arrangement was viewed as very unfair by anti-slavery politicians and a wide group of their supporters, they had lower reputational costs if they used legislative disruption. As we will see, they had electoral incentives to push the boundaries of accepted legislative rules. In sum, they had far more to gain from breaking and changing the rules than following them.

\cite{Pierson1995} argues that Republican and anti-slavery proponent Charles Sumner was doing just this when he gave his May 1856 ``The Crime Against Kansas'' speech in the Senate. The speech broadly concerned the need to admit Kansas as a free state. Admitting Kansas as a free state would have resulted in a Senate that was majority free state for the first time.\footnote{California was admitted as a free state in 1850, but committed to sending one pro and one anti-slavery senator to Congress.} The speech included a number of personal attacks on pro-slavery senators, multiple allusions to slavery as rape, and denunciations of ``slave power''. This speech only barely stayed within the Senate's rules of polite discourse, especially its prohibitions of discussions of sex (a number of Democratic senators argued that it actually had broken these rules). Pierson contends that, rather than being purely emotional and personal, the speech was the culmination of an electorally motivated strategy to push the boundaries of the Senate's rules. Early in his first Senate term beginning in 1851 Sumner spoke ``within the rhetorical restraints imposed by his minority position in the Senate as well as the minority status of his own [Free Soil then Republican] party'' \cite[534]{Pierson1995}. In the early to mid-1850s, party alignments shifted considerably, especially following the collapse of the Whig party in 1852.\footnote{The Whigs had been divided on the issue of slavery} The new Republican and anti-Catholic Know Nothing parties were left to compete for the allegiance of Northern voters who were both anti-Catholic and anti-slave power. Sumner attempted to appeal to these voters by rhetorically emphasizing the problem of disproportionate slave power. He increased his rhetoric, pushing the bounds of Senate rules to gain more publicity for the Republican's anti-slave power position.

Representative Preston Brooks, the nephew of the senator Sumner had insulted in the speech, caned Charles Sumner. Why? The Senate's disproportionality not only made it difficult for minority anti-slavery senators to commit to the rules, it also made it difficult for the majority pro-slavery Democrats as well. They clearly did not view Republican's commitments to the status quo as credible. They had a lot to lose from changes that would increase Senate proportionality, as this likely meant an end to slavery. As such, pro-slavery advocates were ``deeply concerned about the security of their `property and their institutions' within the Union'' \cite[281]{Mittal2013}.

Rather than being a personal attack, Pierson argues that Preston Brooks' caning was ``approved by most of the Democratic party both in anticipation of and following the attack'' \cite[553]{Pierson1995}. The Democratic party's denunciation of Sumner's speech on moral grounds (especially the allusions to rape) and Brooks' subsequent attack on Sumner \emph{three days after the speech} were ``designed to halt an escalation of anti-slavery rhetoric'' in the press \cite[553]{Pierson1995}. Rather than suffering reputational damage, Sumner and Brooks gained popularity among their supporters. For example, apparently ``ladies of the South would send [Brooks] hickory sticks, with which to chastise Abolitionists'' \cite[255]{Donald2009}.

Senator Sumner's rule stretching speech and his subsequent caning happened because of an inability of the two sides to commit to follow established Senate rules and peacefully bargain on new ones. Because of increasing disproportionality caused by exogenous shocks both sides had more to gain from disruption and violence.

It is unclear if, at all, the size of the legislative majority, legislative immunity, or culture played a role in this incident. The pro-slavery Democrats controlled 63\% of the Senate. This is certainly not a minority, which theoretically is most strongly indicative of a tendency for violence. Brooks was arrested for beating Sumner indicating that a formal lack of immunity from arrest did not stop him from using violence.\footnote{Though arrested, Brook's ultimate punishment was a \$300 fine \cite[59]{Wolfe2004}.} It is difficult to tell how much cultural values may have played a role in the caning. Certainly it appears that the senators involved on both sides had a low respect for authority. However, this appears to be more the result of increasing disproportionality and credible commitment problems than an independent cause of violence.

%%%%%%%%%%%%%% Describing violence in National Legislatures
\section*{Describing Violence in National Legislative Chambers Around the World (1981-2012)}

In order to systematically explore the causes of legislative violence across the contemporary globe, I used keyword searches of the Google News Archive, LexisNexis, NewsLibrary, NewsBank, general Google Search, and YouTube to create a data set of {\emph{physical fights between legislators in national legislative chambers}}. Keywords included ``parliament'', ``legislature'', ``national assembly'', ``brawls'', ``scuffles'', and ``fights''. See the Online Appendix for further details. These searches were supplemented by expert information from colleagues resulting in a data set of 131 incidents of legislative violence between 1981 and 2012. 86 of which were in 30 democracies.

We can see in Figure \ref{leg_map} that these events have occurred in many regions around the world. They do not appear to be confined to any one cultural group or region, as a simple regional culture explanation would predict. Violence is nonetheless not evenly distributed across countries as we might expect if it was purely the result of legislators with violent personalities. Although I observed 30 democracies having legislative violence, about 60\% of these fights occurred in seven countries with four or more legislative brawls. These countries are: India, Italy, South Korea, Mexico, Taiwan, Turkey, and Ukraine.

Before moving on to the regression analysis, it is useful to first examine the simple associations between proportional electoral outcomes, democratic age, and violence in this data. Figure \ref{framework_empirical} plots variables measuring these concepts in the entire sample of countries. In the following parametric analysis we will only look at democratic legislatures. Each point represents a country-year. It's notable that virtually every observed incident of violence took place in legislatures with more disproportionate seat distributions. Similarly, older democracies (approximately 55 years or older) were never observed having legislative brawls in the chamber. See below for details about how disproportionality and democratic age are measured.

\vspace{0.5cm}

\textbf{FIGURE 1 ABOUT HERE}

\vspace{0.5cm}

\textbf{FIGURE 2 ABOUT HERE}

\vspace{0.5cm}

%%%%%%%%%%%%%%%%%%%%%%%%%%%%%%%%%%%%%%%%%%%%%%%%%%%%%%%%%%%%%%%%%%%%%%%%%%% Empirical Analysis
\section*{Regression Models: Set up}

To more closely and robustly investigate these findings I use incidents of legislative brawls in democratic legislative chambers per country-year\footnote{Country-years with more than one act of violence are given multiple records. See below for how standard errors were adjusted to address statistical issues related to this.} as the dependent variable in a series of regression models. All country-years for which data is available over the observation period, regardless of legislative violence, are included in the sample. Country-years were deemed to be democratic if their Polity IV score \citep{Marshall2009} was greater than five, the point above which that data set's creators classify a country as democratic.

Legislative brawls are rare. Most of the time the overwhelming majority of legislatures do not have physical fights. The rarity of legislative brawls creates some statistical problems. Standard logistic regression techniques can ``sharply underestimate the probability of rare events" \cite[137]{KingRareEventsPA2001}. Estimated regression coefficients from logistic regression analysis with many fewer observed events than non-events will be too small. Furthermore, standard methods for computing event probabilities with logistic regression produce results biased in the same direction as the coefficient estimates. To overcome this problem \cite{KingRareEvents2001,KingRareEventsPA2001} propose a bias-corrected logistic model for rare events data--rare events logistic regression. I use this method below.

Individual observations are clearly correlated within countries and years, especially when there were multiple acts of violence in a country in a year. To address this issue I used robust standard errors \citep{Golder2006, Mainwaring2007}. Standard errors were adjusted using \cite{Lumley1999} weighted empirical adaptive variance estimators (WEAVE) where the true dependence structure does not need to be specified prior to running the analysis.

%%%%%%%%%%%%%%%%%%%%%%%%%%%%%%%%%%%%%%%%%%%% Independent variables
\subsection*{Right-hand Variables}

Variable descriptions and sources are summarized below, as well as in a table that can be found in the Online Appendix. A matrix illustrating the variables' correlations can also be found in the Online Appendix. This figure includes the variables' observed minimum and maximum values for reference.

I measure the {\emph{age of a democratic regime}} as the number of years a country's Polity IV score \citep{Marshall2009} is continuously greater than five. Because this variable is highly right-skewed, I transformed it using the natural logarithm. As noted earlier, simply looking at the electoral mechanics confuses mechanisms with outcomes. So, I use the standard Least Squares or Gallagher Index \citep{Gallagher1991} to measure realized {\emph{electoral disproportionality}}. The available data is in terms of overall national disproportionality. To gain maximum coverage, I compiled the data from both \cite{Gallagher2012} and \cite{Carey2011}. Full details can be found at: [WITHHELD FOR BLIND REVIEW]
% \url{http://christophergandrud.github.com/Disproportionality_Data/}
.
A country's disproportionality score is treated as constant from the year of an election until the year before the following election. Higher values on the Gallagher Index indicate more disproportionate electoral outcomes.

As we saw in Figure \ref{framework_empirical} there appears to be a strong negative correlation between very low levels of disproportionality and legislative violence. Only two instances of violence were observed in countries with a disproportionality score of less than about 2.5.\footnote{Though the observed range of disproportionality scores goes from almost 0 to 30, countries with scores less than 2.5 are not outliers. About 16\% of county-years in the full sample had scores of 2.5 or smaller.} This can be explained by fairness equilibria and is partially corroborated by Marien's \citeyearpar{Marien2011} finding regarding very low disproportionality measured with the Gallagher Index and political trust discussed above. To capture a possible disproportionality threshold effect--where low disproportionality is associated with stronger feelings of fairness--I created a low disproportionality dummy variable. County-years with disproportionality greater than or equal to the observed median in the full sample (6.34)\footnote{The observed range is 0.26 to 34.52.} are coded as having higher disproportionality and those with scores lower than 6.34 are coded as having lower disproportionality.\footnote{Medians of other sample subsets, such as all country-years with democratic legislatures, were also examined. The results were substantively similar. Additional models were estimated with the continuous disproportionality measure. The parameter estimates were not statistically significant and are not shown.}

I also investigated the possibility that a number of political, institutional, and cultural variables may be associated with legislative violence and that the key variables from my argument have spurious associations. To get a sense of how the size of the governing majority is associated with legislative violence I include the government {\emph{majority}} variable (from the DPI). It simply measures the seats held by governing parties as a proportion of all seats. I transformed the variable from a proportion to a percentage to ease interpretation. Legislators may be less likely to attack one another if they know that they could be arrested for assault. To examine this possibility I include Fish and Koening's \citeyearpar{Fish2009} dichotomous \emph{legislator immunity} variable. It equals one if national legislators are immune from arrest and/or prosecution and zero otherwise. Unfortunately, their data only captures legislative immunity in 2007. I extrapolated the 2007 value of the variable to the other observation years.\footnote{I also considered using the legislator immunity variable from the Comparative Constitutions Project \citep{ElkinsCCP2010}. However, the coding of the immunity variable is vaguer than Fish and Koening's and only deals with constitutionally mandated immunity.} We should therefore approach results from this variable with caution since it might not be a valid indicator for all country-years.

To assess any effect of coalition compared to single-party governments I included a transformation of the DPI {\emph{government fractionalization}} variable. It is the probability that two randomly picked deputies in the government are from different parties. I used the fractionalization variable to create an indicator of {\emph{single-party}} government. It is simply a dummy equaling one if fractionalization was zero, i.e. all governing legislators were from the same party. In general, single party governments probably are better able to pass policies very close to their ideal preferences. This could heighten losers' losses and make them less likely to want to conform to non-violent legislative rules. However, all single party governments are not created equal. Some parties may act as umbrella parties that incorporate many different factions. Others are narrowly focused on a particular constituency.

Perhaps legislators are more violent if they become frustrated with their (in)ability to effect policy change. Control over legislative resources typically closely corresponds to an ability to shape outcomes. However, veto players outside the legislature may severely restrict which bills pass the legislature are actually enacted. To examine this possibility, I included Henisz's \citeyearpar{Henisz2004} measure of political constraints.\footnote{The version of the variable included is POLCONIII. Results are similar for the other version--POLCONV--which includes two additional veto points: the judiciary and sub-federal units.} This variable captures the feasibility of policy change. It ranges between zero and one. Higher values indicate that there are more veto players with disparate preferences and thus there is a lower likelihood of policy change. The variable has been updated through 2011.

To examine the possibility that armed conflict may be associated with legislative violence I created a binary internal armed conflict variable using the UCDP/PRIO Armed Conflict Dataset \citep{Themner2014}. The variable is one if there is an internal armed conflict (either involving or not involving external states) and zero otherwise.

To examine relationships between societal-level values and legislative violence I rely primarily on data from the World Values Survey \citeyearpar{WVS2009}. Over the course of his research, Ingelhart found that his composite {\emph{self-expression}} indicator is the best way to capture cultural and normative differences between democracies and non-democracies. Societies have high self-expression scores if they emphasize ``liberty and participation, public self-expression, tolerance of diversity, interpersonal trust, and life satisfaction" \citep[64]{Inglehart2003}. So I include the self-expression variable from the World Values Survey.\footnote{In earlier versions of the models I also included components of this variable. However, they were either never statistically significant or produced unintelligible and unstable results.} Following \cite{Inglehart2003}, I average the variables across individual participants within countries and survey waves. I only use the third through fifth survey waves\footnote{The surveys were taken in the following years: \\ Wave 3: 1994--1998 \\ Wave 4: 1999--2004 \\ Wave 5: 2005--2007} as the first two waves have very poor coverage. I used wave 3 for all years before 1998, wave 4 for all years between 1999 and 2004 and wave 5 onward.

In addition, competition in more ethnically divided societies may be generally more intense. These conflicts may spill over into legislatures where they precipitate violence between members. I include Alesina et al.'s \citeyearpar{Alesina2003} {\emph{ethnic fractionalization}} data to account for the fact that a legislature's composition in terms of its fractionalization is not only a function of political institutions, but also social divisions \citep{Neto1997, Mozaffar2003}. The variable measures the probability that two randomly selected members of society will be from different ethnic groups. Higher values indicate more fractionalization.

Please see the Online Appendix for discussions of and results for a number of other variables that were tested to examine the possibility of omitted variable bias. These variables include: the percentage of women in parliament, murder rates, incentives for legislators to cultivate a personal vote, the raw government fractionalization variable, the effective number of parties in the legislature, federalism, the Gini coefficient, and GDP per capita.

%%%%%%%%%%%%%%%%%%%%%%%%%%%%%%%%%%%%%%%%%%%% Analyses


\section*{Regression Models: Results}

I used the {\tt{relogit}} model from the {\tt{R}} package {\tt{Zelig}} \citep{IMAIKingZelig2008} to estimate regression models. To get a better understanding of the magnitude of the estimated relationships between the right-hand variables and violence beyond simple statistical significance \citep{ward2010perils} I also used Zelig to predict incident probabilities with 1,000 simulations per fitted value \citep[]{King2002}. Expected probabilities for various fitted values of the key findings of interest are shown in Figure \ref{pred_prob}. Note: because the variables are in country-year records, all of the results should be interpreted in terms of the predicted effect of a variable on the probability of legislative violence in a given country per year.

I observed relatively few incidents of violence in the 1980s. There were only 8 observed incidents before 1990 in the full sample. To examine any estimation biases this might create I ran the regressions on a further constricted sample of democratic legislatures from 1990. The results were broadly similar across the two samples. Regression coefficient point estimates and robust standard errors for the sample that is truncated from 1990 can be found in Table \ref{outputTable.1990}. Results from the full observation period can be found in the Online Appendix. In addition, the Online Appendix contains an examination of why more violent incidents are observed in more recent time periods.

\vspace{0.5cm}

\textbf{TABLE I ABOUT HERE}

\vspace{0.5cm}

\textbf{FIGURE 3 ABOUT HERE}

\vspace{0.5cm}


\paragraph{Disproportionality}

Across virtually all of the models the estimates for the dummy {\emph{low disproportionality}} variable indicate that proportional electoral outcomes are associated with less legislative violence. This finding is robust at least at the 5\% significance level in most model specifications.\footnote{The only model in Table \ref{outputTable.1990} where it drops just below the 10\% significance level is Model 6. This model includes the self expression variable which, due to list-wise deletion for missing-ness, causes the sample size to be cut almost in half. In the full 1990-2012 sample the coefficient estimate was significant at the 10\% level with self-expression included. Please see the Online Appendix for details.} The finding corroborates what we saw in the Sumner caning case study and Figure \ref{framework_empirical}. More interesting than simple statistical significance or coefficient point estimates is the magnitude of and uncertainty surrounding the disproportionality/violence relationship. To get a sense of the magnitude of this estimated relationship and the uncertainty around it, I plotted the predicted probabilities of having a legislative brawl in the left-most plot of Figure \ref{pred_prob} using estimates from Model 3. We can see that the median of the expected probabilities of a country having violence when disproportionality is greater than or equal to 6.34 is approximately three percent in a given year. All other variables are fitted at their means. Countries with disproportionality less than 6.34 have a median expected probability closer to one percent.

\paragraph{Age of democracy}

Also, corroborating the evidence we saw before, the analyses indicate that older democracies tend to have less legislative violence. This result is significant at least at the 5\% level in all of the models. Looking at the middle-panel of Figure \ref{pred_prob} we can see that violence is more likely in younger democracies. Very young democracies are predicted to have a well over five percent probability of experiencing legislative violence in a given year. The probability of violence decreases steadily as a democracy ages.

\paragraph{Governing majorities}

I also found a negative relationship between the size of governments' legislative majorities and violence. We can see in the right-most panel of Figure \ref{pred_prob} that the predicted probability of violence in countries with minority governments is relatively high at about 5\% in a year. This finding fits well within this article's main argument. Minority governments' control of the agenda and thus policy outcomes is disproportionate--even if the minority government has a constrained ability to affect policy change--and may see considerable opportunities to shift power in their direction. They would be less able to make credible commitments. The minority government may at the same time use violence to try to block attempts to increase proportionality. This could make it more difficult for both sides to credibly commit.

It is still less theoretically clear what is causing the very low estimated probability of violence in large majority legislatures. It could be that legislators in these parliaments view the decisions of the larger majority as more legitimate or it could be that hegemonic parties are able to effectively quash disruption and violence before it starts. It is difficult to separate out these two possible causes here. I did try to rule out the possibility that the result is being driven by legislatures with very powerful parties that control virtually all of the seats. To do this I reran the models where observations with government majorities greater than or equal to 75\% were dropped. The results (not shown) nonetheless persisted. Further case study work is needed to understand the causes of a lack of violence in legislatures with large majorities.

\paragraph{Societal-level variables}

None of the cultural or ethnic fractionalization variables were found to be associated with legislative violence. We should be somewhat sceptical about the strength of the conclusions we can draw from the self-expression variable results. As mentioned earlier there might be a highly endogenous relationship between culture and institutions. However, if societal-level culture was driving institutions that were associated with legislative violence, presumably the cultural variables would have also been associated with violence in models without the institutional variables, which they were not (see the Online Appendix). Nonetheless, it takes a bit of a leap to believe that the mean level of self-expression found using a national-level survey accurately reflects the values held by elite individuals in legislatures. Further work is needed to make stronger conclusions about the relationships between culture and the propensity for legislative violence. This research could possibly use individual legislator-level surveys that would allow us to directly measure the distribution of values among actual legislators. In addition, we did not find robust evidence that internal armed conflict is associated with legislative violence. At this point we can say that we have not yet found evidence that societal-level factors are \emph{directly} associated with a propensity for violence. Though, as the case study indicates, they can affect factors such as disproportionality which in turn create credible commitment problems and violence.

\paragraph{Other political and institutional variables}

Results for other political and institutional variables were largely not statistically significant. Legislative immunity from arrest and/or prosecution is not significantly associated with legislative violence. We should approach this result cautiously since the Fish and Koening immunity variable is based on observations in 2007 and therefore might not be a valid indicator for many country-years. The single-party government variable was not statistically significant in the analyses. Finally, political constraints were not found to be linearly associated with legislative brawls. All of these variables, and others discussed in the Online Appendix, are not as directly theoretically related to an ability to make credible legislative commitments, compared to disproportionality, democratic age and, to a lesser extent, governing majority size. So it should not come as too much of a surprise to find that they are not robustly associated with legislative violence.

\paragraph{Additional robustness checks and interactive models}

Please see the Online Appendix for a variety of other models that check for omitted variable bias and the possibility of interactive relationships. Overall, the core findings presented thus far are unchanged in these models. There is some evidence that factors such as ethnic fractionalization and political constraints could mediate the credible commitment problems in new democracies; less fractionalized societies and governments with higher constraints may have less violence in new democracies because these factors help overcome credible commitment problems. However, the substantive significance of these interactions is generally weak when we examine expected probabilities for a range of meaningful fitted values. As such, these results should be taken as very preliminary evidence that perhaps the credible commitment problems of new democracies are less severe in certain societies and governments with strong constraints on the ability to alter policy. Given the small number of observed incidents of legislative violence there could simply not be enough information in the data to draw meaningful conclusions about these types of interactive relationship at this time \citep[]{Brambor2006}.

%%%%%%%%%%%%%%%%%%%%%%%%%%%%%%%%%%%%%%%%%%%% Conclusions
\section*{Conclusions: What Keeps Legislators Two Sword Lengths Apart?}

In this article--the first to systematically examine legislative violence in democratic legislative chambers on a global scale--I developed and tested an argument that violence in these chambers is more likely when legislators are unable to make credible commitments to peaceful bargaining outcomes. What conclusions can we make from the article's findings about why legislators are kept ``two sword lengths'' apart (or not) and what implications do they have?

The article's findings suggest that countries with highly proportional electoral outcomes rarely experience legislative violence. Looking broadly at the contemporary global sample, the relationship appears to be subject to a threshold effect where countries with Gallagher Index scores below about 6.34 rarely experience legislative violence and there were only two observed incidents when the Index was below 2.5. This finding is perhaps more important than the article's age of democracy result for democratic institutional designers who want to actively keep legislators two sword lengths apart. Democratic age is subject to many factors far outside of democratic planners' control. Electoral disproportionality is much more malleable. Proportional electoral systems are the most obvious tool electoral system designers have to increase the correspondence between party's votes and seats \citep{Carey2011}. Depending on the distribution of preferences in the electorate, these systems can be tweaked by increasing district magnitude or altering the formula used to translate votes into seats. The findings in this article suggest that electoral system designers may want to aim for very proportional outcomes, to the extent possible, when a country transitions to democracy in order to prevent violence in the legislature.

It is important to note that a lack of violence is not \emph{necessarily} an indication of a well-functioning democratic institution. A legislature may be peaceful because the governing party so thoroughly controls legislative power and oppresses opposition that opposing sides are incapable of truely bargaining, even through violence. This may be the case in legislatures with very large majorities. However, the presence of violence \emph{does} indicated incomplete and dysfunctional democratic institutions. Democracy aims to replace ``violent confrontation [with] debate and discussion, aspiring to the peaceful reconciliation of the conflict and difference which are inherent in any modern complex society'' \cite[220]{Schwarzmantel2010}. A goal of this article was to find and highlight the institutional factors that could help move such legislatures closer to the democratic ideal.

As this is the first large scale study on this subject there is still considerable work that can be done to better understand the causes and consequences of legislative violence. As mentioned earlier, more work is needed on legislator-level cultural values and the effects of very large majority governments. Research could explore how societal changes, such as demographic and resulting party system changes affect factors such as disproportionately, leading to violence. Further case study work could examine non-electoral fairness issues that could exacerbate credible commitment problems. In particular, more research could be done on post-election solutions that make the distribution of legislative resources more proportional \cite[for example, see][who examined informal minority party access to power in Japan's Diet]{Wolfe2004}. Non-disproportionality/shifts in power causes of credible commitment problems should also be explored. Another avenue worth pursuing would be to empirically examine how credible commitment problems change the likelihood of less severe forms of legislative disruption, such as boycotts and shouting. Finally, what are the consequences of legislative violence, especially for citizens' perceptions of democratic legitimacy and, indeed, the long-term viability of democratic regimes?

\subsection*{Replication}

\noindent Full replication files--including data and source code--for the large-n analysis in this article can be found at
%\url{https://github.com/christophergandrud/leg_violence_paper1}
[URL WITHHELD FOR BLIND REVIEW]. All analyses were conducted using \texttt{R} \citep{R-cite}.


%%%%%%%%%%%%%%%%%%%%%%%%%%%%%%%%%%%%%%% Appendix

\bibliographystyle{apsr}
\bibliography{LegViolence}

\subsection*{Bibliographic Statement}

[WITHHELD FOR BLIND REVIEW]
% \noindent CHRISTOPHER GANDRUD, b, 1984; PhD in Political Science (London School of Economics, 2012); Post-doctoral Fellow, Hertie School of Governance (2013-- ); current main interests: financial policymaking and legislative politics. Various recent publications in journals such as the \emph{Journal of Common Market Studies}, \emph{Political Science Research and Methods}, and \emph{Review of International Political Economy}.

\clearpage

%%%%%%%% Map of Incidences
\begin{figure}[h!]
    \centering

        %% Created with Analysis/main_analysis_1.R
        \includegraphics[width = 13cm]{incidence_map.png}

        \caption{Incidences of Physical Fights Between Legislators in National Legislative Chambers (1981-2012)}
        \label{leg_map}

\end{figure}

%%%%%%%%%%%%%%%% Scatterplot of Disproportionality, age_dem, and Violence %%%%%%%%%%%%%%%%%%%%%%%%
\begin{figure}[t]
    \begin{center}

\begin{knitrout}
\definecolor{shadecolor}{rgb}{0.969, 0.969, 0.969}\color{fgcolor}
\includegraphics[width=0.8\linewidth]{figure/FrameworkEmpirical-1} 

\end{knitrout}
    \end{center}

    \caption{Scatter plots of Disproportionality, Age of Democracy, and Violence in the National Legislatures.}
    \label{framework_empirical}

    \begin{singlespace}
        {\scriptsize{Each point represents a country-year. The data is from a sample of countries from 1981 through 2012 due to data availability. The points are jittered horizontally.}}
    \end{singlespace}

\end{figure}

%%%%%%%%%%%%%%%% Run Analyses %%%%%%%%%%%%%%%%%%%%%%%%


%%%%%%%% Elected Legislatures Results Table from 1990
\begin{table}
\caption{Legislative Violence Regression Results (Democratic Legislatures 1990-2012)}
\label{outputTable.1990}
\begin{center}
\scalebox{0.8}{

% Table created by stargazer v.5.1 by Marek Hlavac, Harvard University. E-mail: hlavac at fas.harvard.edu
% Date and time: Mon, May 04, 2015 - 14:35:50
\begin{tabular}{@{\extracolsep{5pt}}lccccccc} 
\\[-1.8ex]\hline 
\hline \\[-1.8ex] 
 & \multicolumn{7}{c}{\textit{Dependent variable:}} \\ 
\cline{2-8} 
\\[-1.8ex] & \multicolumn{7}{c}{Violent Incident} \\ 
\\[-1.8ex] & (1) & (2) & (3) & (4) & (5) & (6) & (7)\\ 
\hline \\[-1.8ex] 
 Lower Disproportionality & $-$0.578$^{**}$ & $-$0.583$^{**}$ & $-$0.586$^{**}$ & $-$0.575$^{**}$ & $-$0.560$^{**}$ & $-$0.467 & $-$0.573$^{**}$ \\ 
  & (0.269) & (0.271) & (0.271) & (0.272) & (0.271) & (0.300) & (0.271) \\ 
  & & & & & & & \\ 
 Dem. Age (log) & $-$0.275$^{**}$ & $-$0.331$^{***}$ & $-$0.329$^{***}$ & $-$0.331$^{***}$ & $-$0.325$^{***}$ & $-$0.409$^{***}$ & $-$0.343$^{***}$ \\ 
  & (0.108) & (0.112) & (0.111) & (0.112) & (0.111) & (0.135) & (0.114) \\ 
  & & & & & & & \\ 
 Majority Size &  & $-$0.022$^{***}$ & $-$0.022$^{**}$ & $-$0.021$^{**}$ & $-$0.021$^{**}$ & $-$0.019$^{*}$ & $-$0.022$^{**}$ \\ 
  &  & (0.009) & (0.008) & (0.009) & (0.008) & (0.010) & (0.009) \\ 
  & & & & & & & \\ 
 Internal Armed Conflict &  &  & 0.440 & 0.421 & 0.413 & 0.386 & 0.449 \\ 
  &  &  & (0.344) & (0.347) & (0.344) & (0.386) & (0.345) \\ 
  & & & & & & & \\ 
 Leg. Immunity &  &  &  & 0.012 &  &  &  \\ 
  &  &  &  & (0.267) &  &  &  \\ 
  & & & & & & & \\ 
 Single Party Gov. &  &  &  & $-$0.049 &  &  &  \\ 
  &  &  &  & (0.263) &  &  &  \\ 
  & & & & & & & \\ 
 Political Constraints &  &  &  &  & $-$0.534 &  &  \\ 
  &  &  &  &  & (0.913) &  &  \\ 
  & & & & & & & \\ 
 Self Expression &  &  &  &  &  & 3.870 &  \\ 
  &  &  &  &  &  & (2.533) &  \\ 
  & & & & & & & \\ 
 Ethnic Frac. &  &  &  &  &  &  & $-$0.308 \\ 
  &  &  &  &  &  &  & (0.614) \\ 
  & & & & & & & \\ 
 (Intercept) & $-$2.120$^{***}$ & $-$0.735 & $-$0.807 & $-$0.811 & $-$0.671 & $-$5.675$^{*}$ & $-$0.659 \\ 
  & (0.307) & (0.559) & (0.561) & (0.655) & (0.625) & (3.180) & (0.645) \\ 
  & & & & & & & \\ 
\hline \\[-1.8ex] 
Observations & 1,492 & 1,441 & 1,441 & 1,419 & 1,417 & 810 & 1,435 \\ 
Log Likelihood & $-$260.703 & $-$254.213 & $-$253.577 & $-$252.777 & $-$252.611 & $-$188.719 & $-$253.279 \\ 
Akaike Inf. Crit. & 527.406 & 516.426 & 517.155 & 519.554 & 517.222 & 389.438 & 518.559 \\ 
\hline 
\hline \\[-1.8ex] 
\multicolumn{8}{l}{$^{*}$p$<$0.1; $^{**}$p$<$0.05; $^{***}$p$<$0.01} \\ 
\multicolumn{8}{l}{Standard errors are in parentheses. All models use robust (WEAVE) standard errors.} \\ 
\end{tabular} 

}
\end{center}
\end{table}


%%%%%%%%%%%%%%%% Expected Probability Graphs %%%%%%%%%%%%%%%%%%%%%%%%
\begin{figure}[t]
    \begin{center}

\begin{knitrout}
\definecolor{shadecolor}{rgb}{0.969, 0.969, 0.969}\color{fgcolor}
\includegraphics[width=0.95\linewidth]{figure/predProb-1} 

\end{knitrout}
    \end{center}
    \caption{Expected Probability of Legislative Violence in Democratic Legislatures per Year}
    \label{pred_prob}
    \begin{singlespace}
      {\scriptsize{The graphs show the median and middle 95\% of 1000 simulations at each fitted value of the variables. The simulations use estimates from Model 3, Table \ref{outputTable.1990}. For each set of simulations all other variables were fitted at their means.}}
    \end{singlespace}
\end{figure}

\end{document}
